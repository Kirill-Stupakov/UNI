\documentclass[11pt, letter, utf-8]{article}

\usepackage[utf8]{inputenc}
\usepackage[russian]{babel}
\usepackage{amsmath, amsmath, amssymb, amsthm}
\usepackage[margin=0.5in]{geometry}
\usepackage[makeroom]{cancel}
\usepackage{hyperref}


\pagestyle{empty}
\begin{document}
    \begin{align*}
        \Delta_n &= 
        \begin{vmatrix}
            \alpha + \beta & \alpha \beta & 0 & 0 & \cdots & 0 \\
            1 & \alpha + \beta & \alpha \beta & 0 & \cdots & 0 \\
            0 & 1 & \alpha + \beta & \alpha \beta & \cdots & 0 \\
            0 & 0 & 1 & \alpha + \beta & \cdots & 0\\
            \vdots & \vdots & \vdots & \vdots & \ddots & \vdots\\
            0 & 0 & 0 & 0 & \cdots & \alpha + \beta 
        \end{vmatrix} = \\
        &=
        (\alpha + \beta)
        \begin{vmatrix}
            \alpha + \beta & \alpha \beta & 0 & \cdots & 0 \\
            1 & \alpha + \beta & \alpha \beta & \cdots & 0 \\
            0 & 1 & \alpha + \beta & \cdots & 0 \\
            \vdots & \vdots & \vdots & \ddots & \vdots\\
            0 & 0 & 0 & \cdots & \alpha + \beta 
        \end{vmatrix}
        - 1 \begin{vmatrix}
            \alpha \beta & 0 & 0 & \cdots & 0 \\
            1 & \alpha + \beta & \alpha \beta & \cdots & 0 \\
            0 & 1 & \alpha + \beta & \cdots & 0 \\
            \vdots & \vdots & \vdots & \ddots & \vdots \\
            0 & 0 & 0 & \cdots & \alpha + \beta
        \end{vmatrix} = \\
        &= (\alpha + \beta) \Delta_{n-1} - \alpha \beta \begin{vmatrix}
            \alpha + \beta & \alpha \beta & \cdots & 0 \\
            1 & \alpha + \beta & \cdots & 0 \\
            \vdots & \vdots & \ddots & \vdots \\
            0 & 0 & \cdots & \alpha + \beta 
        \end{vmatrix}
        + 1 \begin{vmatrix}
            0 & 0 & \cdots & 0 \\
            1 & \alpha + \beta & \cdots & 0 \\
            \vdots & \vdots & \ddots & \vdots \\
            0 & 0 & \cdots & \alpha + \beta
        \end{vmatrix} = \\
        &= (\alpha + \beta) \Delta_{n-1} - \alpha \beta \Delta_{n-2}
    \end{align*}
    Получили рекурсивную формулу:
    \[\Delta_{n} = (\alpha + \beta) \Delta_{n-1} - \alpha \beta \Delta_{n-2}\]
    Посчитаем первые несколько значений:
    \begin{align*}
        \Delta _1 &= \begin{vmatrix}
            \alpha + \beta
        \end{vmatrix} = \alpha + \beta \\
        \Delta _2 &= \begin{vmatrix}
            \alpha + \beta & \alpha \beta \\
            1 & \alpha + \beta
        \end{vmatrix} = (\alpha + \beta)^2 - \alpha \beta = \alpha ^ 2 + \alpha \beta + \beta ^2 \\
        \Delta _3 &= (\alpha + \beta)\Delta_2 - \alpha \beta \Delta_1 = \alpha ^3 + \alpha ^2 \beta + \alpha \beta^2 + \beta^3
    \end{align*}
    Легко угадать общую формулу:
    \[\Delta_{n} = \sum_{i = 1}^n \alpha ^i \beta ^{n-i}\]
    Докажем её по индукции. База: $n = 1$ и $n = 2$.\\
    Предположение:
    \[\Delta_{n-2} =\sum_{i = 0}^{n-2} \alpha^i \beta^{n-2-i} \text{ и } \Delta_{n-1} =\sum_{i = 0}^{n-1} \alpha^i \beta^{n-1-i}\]
    Переход:
    \begin{align*}
        \Delta_n &= (\alpha + \beta) \sum_{i = 0}^{n-1}\alpha^i \beta^{n-1-i} - \alpha \beta \sum_{i = 0}^{n-2}\alpha^i \beta^{n-2-i} = \\
        &= \alpha \sum_{i = 0}^{n-1}\alpha^i \beta^{n-1-i} + \beta \sum_{i = 0}^{n-1}\alpha^i \beta^{n-1-i} - \alpha \beta \sum_{i = 0}^{n-2}\alpha^i \beta^{n-2-i} = \\
        &= \sum_{i = 0}^{n-1}\alpha^{i+1} \beta^{n-1-i} + \sum_{i = 0}^{n-1}\alpha^i \beta^{n-i} - \sum_{i = 0}^{n-2}\alpha^{i+1} \beta^{n-1-i} = \\
        &= \sum_{j = i+1 = 0}^{n}\alpha^{j} \beta^{n-j} - \cancel{\alpha^0 \beta^n} + \cancel{\sum_{i = 0}^{n-1}\alpha^i \beta^{n-i}} - \cancel{\sum_{j = i+1 = 0}^{n-1}\alpha^{j} \beta^{n-j}} + \cancel{\alpha^0 \beta^n} = \\
        &= \sum_{j = i+1 = 0}^{n}\alpha^{j} \beta^{n-j} \qed
    \end{align*}
\end{document}